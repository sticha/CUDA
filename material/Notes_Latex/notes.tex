\documentclass[A4,12pt]{article}
\usepackage{amssymb}
\usepackage{amsmath}
\usepackage{amsthm}
\usepackage[nooneline]{subfigure}
\usepackage{graphicx}
\usepackage{varwidth}
\usepackage{float}
\usepackage{fullpage} 
\usepackage{color}
\usepackage{nicefrac}
\usepackage{epstopdf}

\newtheorem{thm}{Theorem}[section]
\newtheorem{prop}[thm]{Proposition}
\newtheorem{lemma}[thm]{Lemma}
\newtheorem{cor}[thm]{Corollary}
\newtheorem{rem}[thm]{Remark}
\newtheorem{defi}[thm]{Definition}
\newtheorem{example}{Example}

\DeclareMathOperator*{\argmin}{arg\,min}
\DeclareMathOperator{\divop}{div}
\DeclareMathOperator*{\proj}{proj}

\newcommand\M{\mathcal{M}}
\newcommand{\R}{\mathbb{R}}
\newcommand{\Rext}{\mathbb{R} \cup \{ \infty \} }
\newcommand{\rslv}[1]{(I + #1)^{-1}}
\newcommand{\norm}[1]{\Vert #1 \Vert}
\newcommand{\abs}[1]{\vert #1 \vert}
\newcommand\MyGreenBox[1]{%
\colorbox{green}{\begin{varwidth}{\dimexpr\linewidth-2\fboxsep}#1\end{varwidth}}}
\providecommand{\iprod}[2]{\langle#1,#2\rangle}



\title{Joint Super-Resolution and Optical Flow Estimation}
%\author{Thomas M\"ollenhoff, Philipp Kr\"uger, Sebastian K\"umper \and Jonas Sticha}
\date{}

\begin{document}
\maketitle

\section{Energy Functional}
Let $\Omega_l \subset \R$ be a discretized rectangular domain with $w \times h$ pixels and $\Omega_h \subset \R$ a domain with $W \times H$ pixels. Let $f_1, \cdots, f_n : \Omega_l \rightarrow \R^k$ be low resolution color input images with $k$ color channels. We seek to jointly estimate high resolution images $u_1, \cdots, u_n : \Omega_h \rightarrow \R^k$ and optical flow fields $v_1, \cdots, v_{n-1} : \Omega_h \rightarrow \R^2$ from the low resolution images.

We model the problem in terms of energy minimization of the following functional:
\begin{equation}
  E(u, v) = \sum_{i=1}^n \alpha \norm{Au_i - f_i}_{1} + \beta TV(u_i) + \gamma \sum_{i=1}^{n-1} \underset{x \in \Omega_h} \sum || u_i(x) - u_{i+1}(x + v_i(x)) ||_1 + TV(v_i^1) + TV(v_i^2) 
\end{equation}

Here $A : (\Omega_h \rightarrow \R^k) \rightarrow (\Omega_l \rightarrow \R^k)$ is a linear operator which maps a high-resolution image to a low resolution image by blurring it with a gaussian kernel and downsampling it.

$TV(u) := \underset{x \in \Omega} \sum \norm{\nabla u(x)}_2$ denotes the TV regularizer.

\section{Optimization}
Since the energy is hard to minimize jointly in $u$ and $v$ we employ a block-coordinate descent approach:

\begin{equation}
\begin{aligned}
&v^{k+1} = \underset{v} \argmin ~ E(u^{k}, v),\\
&u^{k+1} = \underset{u} \argmin ~ E(u, v^{k+1}).
\end{aligned}
\end{equation}

\subsection{Solving the Problem in $v$ (TV-L1 Optical Flow).}
For fixed $u^k$ the problem reads:
\begin{equation}
  v^{k+1} = \underset{v} \argmin ~ \gamma
\underset{x \in \Omega_h} \sum \sum_{i=1}^{n-1} || u_i^k(x) - u_{i+1}^k(x + v_i(x)) ||_1 + TV(v_i^1) + TV(v_i^2) 
\end{equation}
For simplicity, we first consider the case $n=2$:
\begin{equation}
  v^{k+1} = \underset{v} \argmin ~  
\gamma \underset{x \in \Omega_h} \sum || u_i^k(x) - u_{i+1}^k(x + v_i(x)) ||_1 + TV(v_i^1) + TV(v_i^2) 
\end{equation}
This energy is nonconvex in $v$, due to the first term. Thus we linearize it using the first order Taylor expansion,
$$
|| u_i^k(x) - u_{i+1}^k(x + v_i(x)) ||_1 \approx || u_1^k(x) - u_2^k(x) - \nabla u_2^k(x)^T v_i(x) ||_1, 
$$
and end up at the following convex problem:
\begin{equation}
  v^{k+1} = \underset{v} \argmin ~ \gamma  
\underset{x \in \Omega_h} \sum || \underbrace{u_1^k(x) - u_2^k(x)}_{=:-b(x)} - \underbrace{\nabla u_2^k(x)^T v(x)}_{=:(Av)(x) } ||_1 + TV(v^1) + TV(v^2),
\label{eq:flow}
\end{equation}
where $v=(v^1, v^2)$.
\subsubsection{Primal-Dual Optimization}
Since the energy is non-differentiable, a gradient descent based approach does not work. We employ the primal-dual algorithm described in \cite{Chambolle-Pock-jmiv11,Pock-Chambolle-iccv11} to minimize the energy. First, we rewrite \eqref{eq:flow} as an equivalent saddle-point problem:
\begin{equation}
\min_{v} \max_{p \in C,q_1 \in D,q_2 \in D} ~ \iprod{p}{Av + b} + \iprod{q_1}{\nabla v^1} + \iprod{q_2}{\nabla v^2}
\end{equation}
The update equations for the algorithm then read:
\begin{equation}
\begin{aligned}
p^{k+1}(x)&=\proj_C(p^k(x) + \sigma_p(x) ((Av^k)(x) + b(x))) \\
q_1^{k+1}(x)&=\proj_D(q_1^k(x) + \sigma_q (\nabla v_1^k)(x)),\\
q_2^{k+1}(x)&=\proj_D(q_2^k(x) + \sigma_q (\nabla v_2^k)(x)),\\
\bar p^{k+1}(x)&=2 p^{k+1} - p^k,\\
\bar q_1^{k+1}(x)&=2 q_1^{k+1} - q_1^k,\\
\bar q_2^{k+1}(x)&=2 q_2^{k+1} - q_2^k,\\
v^{k+1}(x)&=v^k - \tau(x) ((A^T \bar p^{k+1})(x) - (\divop \bar q_1^{k+1})(x) - (\divop \bar q_2^{k+1})(x) ).
\end{aligned}
\end{equation}
The sets $C$ and $D$ are defined as
\begin{equation}
\begin{aligned}
& C = \{x \in \R ~|~ |x| \leq \gamma \}, \\
& D = \{x \in \R^{2 n_c} ~|~ \norm{x}_2 \leq 1 \}, 
\end{aligned}
\end{equation}
and the projections $\proj_C$, $\proj_D$ can be implemented as an orthogonal projection on a sphere. Only project if you lie outside of the constraint.

The step sizes are chosen according to the scheme described in \cite{Pock-Chambolle-iccv11} (see Lemma 2, equation 10, we set $\alpha=1$):
\begin{equation}
  \begin{aligned}
    &\sigma_p(x) = \frac{1}{\sum_j |A(x,j)|}, \\
    &\sigma_q = \frac{1}{2},\\
    &\tau(x) = \frac{1}{2 + 2 + \sum_i |A(i,x)|},
  \end{aligned}
\end{equation}
where $A(x,j)$ denotes the element in row $x$ and column $j$.

Allocate memory for the variables $p \in \R^{w*h*n_c},q_1 \in \R^{w*h*2*n_c},q_2 \in \R^{w*h*2*n_c},\bar p, \bar q_1, \bar q_2, v \in \R^{w*h*2*n_c}$ as {\tt float} arrays and implement CUDA kernels for the update equations of the primal-dual algorithm. One kernel should perform the update in $p, q_1, q_2$ and do the overrelaxation, the other kernel should do the update in $v$. 

\subsection{Solving the Problem in $u$.}
\begin{equation}
\begin{aligned}
u_i^{k+1}&=u_i^k-\tau_i(A^T\bar p_i^{k+1}-div(\bar q_i^{k+1})+B^T\bar r^{k+1}) , \\
\\
q_i^{k+1}&=\proj_E(q_i^k + \sigma_q (\nabla u_i)), \\
p_i^{k+1}&=\proj_F(p_i^k + \sigma_p (Au_i^k - f_i)), \\
r^{k+1}&=\proj_G(r^{k+1} + \sigma_r B\begin{pmatrix}u_1^k \\ u_2^k\end{pmatrix}), \\
\bar q_i^{k+1}&=2q_i^{k+1}-q_i^k; \quad \bar p_i^{k+1},\quad \textrm{and} \quad \bar r^{k+1} analogously \\
\end{aligned}
\end{equation}
\\
\begin{equation}
\begin{aligned}
\sigma_q &= \frac{1}{2}, \\
\sigma_p &= 1, \\
\tau_i&=\frac{1}{1+4+?(x)}, \\
?(x)&=\begin{cases} 1 & i=1 \\ 1+|(v/u?)^1(x)|*2+|(v/u?)^2(x)|*2 & i \geq 2 \end{cases} \\
\sigma_r (x_{(ijc)}) &= \frac{1}{2+2|(u/v?)^1(x_{(ij)})|+2|(u/v?)^2(x)|} \\
\end{aligned}
\end{equation}
\\
\begin{equation}
\begin{aligned}
B&=\begin{pmatrix}I, &-I-(u/v?)^1\delta_x-(u/v?)^2\delta_y\end{pmatrix}, \\
B^T&=\begin{pmatrix}I \\ -I+\underset{dirichlet^x} {bw\_diff} (u/v?)^1+\underset{dirichlet^y} {bw\_diff} (u/v?)^2\end{pmatrix}, \\
A&=DB, \\
A^T&=BD^T \\
\end{aligned}
\end{equation}
\\
\begin{equation}
\begin{aligned}
E &= \{x \in \R ~|~ \norm{x}_2 \leq \beta \}, \\
F &= \{x \in \R ~|~ |x| \leq \alpha \}, \\
G &= \{x \in \R ~|~ |x| \leq \gamma \} \\
\end{aligned}
\end{equation}
\\
\begin{equation}
\begin{aligned}
\delta_x&=D_x=\begin{pmatrix}1 & -1 & \\ & \ddots & \\ & & \ddots \end{pmatrix}, \\
v_x&=\begin{pmatrix}v_x(?,?)& & & \\ & \ddots & & \\ & & \ddots & \\ & & & v_x(?,?) \end{pmatrix}  \\
\end{aligned}
\end{equation}
\\
\begin{equation}
\begin{aligned}
A&\colon\R^{W*H*c}\mapsto\R^{w*h*c}, \\
B_{flow}&\colon\R^{2*W*H*c}\mapsto\R^{W*H*c}, \\
B_l&\colon\R^{W*H*c}\mapsto\R^{W*H*c}, \\
D&\colon\R^{W*H*c}\mapsto\R^{w*h*c} \\
\end{aligned}
\end{equation}
% For fixed $v^{k+1}$, the energy minimization in $u$ reads:
% \begin{equation}
%   u^{k+1} = \underset{u} \argmin ~ \sum_{i=1}^n \norm{Au_i - f_i}_{2,1} + TV(u_i) + \sum_{i=1}^{n-1} || u_i - B u_{i+1} ||_{2,1},
% \end{equation}
% where $B : (\Omega_h \rightarrow \R^k) \rightarrow (\Omega_h \rightarrow \R^k)$ is a warping operator, which can be constructed from $v^{k+1}$.

\bibliographystyle{plain}
\bibliography{references}


\end{document}
